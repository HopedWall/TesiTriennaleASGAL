\documentclass[a4paper,12pt]{article}
\setlength{\parindent}{0pt}
\usepackage{graphicx}
\usepackage{hyperref}
\usepackage[T1]{fontenc}
\usepackage[utf8]{inputenc}
\usepackage{setspace}
\usepackage[paper=a4paper,margin=1in]{geometry}
%\usepackage[ruled,vlined]{algorithm2e}
\usepackage{algorithm}
\usepackage{algpseudocode}


\begin{document}
	\pagenumbering{gobble}

	\include{./chapters/frontespizio}
    
  \newpage
    
   % Abstract 
   \begin{abstract}
   In questa tesi si discuterà l'estensione di ASGAL (un tool sviluppato dall' AlgoLab in grado di rilevare eventi di Alternative Splicing a partire da campioni di RNA-Seq) per il supporto alle read in formato paired-end. 
   Dopo aver introdotto i concetti necessari per comprendere questo documento, verranno evidenziate le principali modifiche apportate ad ASGAL, ponendo 	  	l'attenzione sulle differenze tra il formato single-end e quello paired-end. Verrà poi mostrato un esempio di funzionamento.
   Infine saranno discussi alcuni possibili sviluppi futuri.
   \end{abstract}
   
   \newpage

		% Tavola dei contenuti
    \tableofcontents
    \newpage
    
    \pagenumbering{arabic}

    	\section{Introduzione}

\subsection{ASGAL}
ASGAL (Alternative Splicing Graph ALigner) \cite{ASGAL} è un tool sviluppato dall'Algolab per l'identificazione di eventi di Alternative Splicing espressi in un campione di RNA-seq a partire da un genoma di riferimento e dall'annotazione di un gene. ASGAL si compone di quattro step:

\begin{enumerate}
	\item \textbf{Costruzione dello splicing graph}: a partire dall'annotazione di un gene e dalla genomica di riferimento, ASGAL costruisce uno splicing graph, ovvero una struttura a grafo che rappresenta tutti i trascritti noti del gene in input. Viene inoltre prodotta una linearizzazione dello splicing graph, che sarà utilizzata in fase di allineamento.
	\item \textbf{Allineamento Splice-Aware}: utilizzando l'algoritmo in \cite{MEM} ASGAL allinea le read di RNA-Seq con la linearizzazione dello splicing graph del gene in input. L'allineamento è Splice-Aware in quanto è necessario tenere traccia della posizione di esoni ed introni per un corretto allineamento.
	\item \textbf{Computazione degli allineamenti Spliced (opzionale)}: gli allineamenti prodotti dall'Allineatore Splice-Aware vengono convertiti nel formato SAM, per permetterne l'elaborazione con strumenti standard.
	\item \textbf{Rilevazione degli eventi di Alternative Splicing}: gli allineamenti prodotti dall'Allineatore Splice-Aware sono analizzati per rilevare gli eventi di Alternative Splicing indotti dalle read del campione. Gli eventi rilevati sono visualizzati in un file csv, che include diverse informazioni su ciascun evento rilevato.
\end{enumerate}

ASGAL è stato sviluppato in C++ e Python: C++ viene utilizzato nella parte di allineamento per la sua efficienza, Python nella parte di rilevazione per la sua semplicità nell'utilizzo di strutture dati complesse. Il progetto è open source, ed è disponibile su GitHub con licenza GNU v3.0. Tutti i componenti sono utilizzabili singolarmente, ma viene fornito uno script principale che ne semplifica l'utilizzo.  

Al momento ASGAL non supporta le read paired-end, un nuovo formato di read prodotte da allineatori NGS (Next Generation Sequencing), che potrebbe portare ad un incremento di efficacia ed efficienza nella rilevazione di eventi di Alternative Splicing. Nei capitoli successivi saranno descritte le principali modifiche ad esso apportate e saranno evidenziate le differenze tra i formati single-end e quelli paired-end.

L'immagine nella pagina successiva riassume il funzionamento di ASGAL.

\begin{figure}[h!]
	\includegraphics[width=\textwidth]{images/asgal.png}
  \caption{Il funzionamento di ASGAL illustrato}
  \label{fig:ASGAL}
\end{figure}

    	\subsection{Alternative Splicing}

L'alternative splicing è un meccanismo utilizzato dalle cellule per produrre proteine (o per meglio dire, \textit{isoforme proteiche}) diverse dallo stesso gene che viene utilizzato da oltre il 75\% dei geni umani \cite{wang2008alternative}. Diversi studi (come \cite{tazi2009alternative} e \cite{rockenstein1995levels}) dimostrano come l'alternative splicing giochi un ruolo fondamentale nello sviluppo di diverse malattie, come ad esempio il cancro o la sindrome di Alzheimer.

Considerando un generico frammento di DNA, esso può essere diviso in esoni (parti codificanti) e introni (parti non codificanti). Durante la fase di Trascrizione gli introni vengono rimossi e la Timina viene trasformata in Uracile, ottenendo pre-mRNA. A questo punto, in un normale processo di Splicing, tutti gli esoni vengono utilizzati, nell'ordine in cui appaiono nel pre-RNA, per ottenere una proteina.

\begin{figure}[h!]
	\centering
	\includegraphics[width=\linewidth]{images/splicing.png}
  \caption{Trascrizione e Splicing}
  \label{fig:Splicing}
\end{figure}

Nel caso di un evento di Alternative Splicing, questo non accade: alcuni esoni potrebbero infatti non essere utilizzati, o apparire in un ordine diverso. Vengono riconosciuti 5 tipi di eventi di Alternative Splicing:

\begin{enumerate}
	\item \textbf{Exon Skipping}: Un esone non appare nel trascritto; quando gli esoni sono più di uno, si parla di \textbf{Multiple Exon Skipping}
	\item \textbf{Mutually Exclusive Exons}: Due esoni non compaiono mai in uno stesso trascritto contemporaneamente
	\item \textbf{Alternative 5' Donor Site}: Parte di un introne nel 5' diventa un esone
	\item \textbf{Alternative 3' Acceptor Site}: Parte di un introne nel 3' diventa un esone
	\item \textbf{Intron Retention}: Parte di un esone diventa un introne
\end{enumerate}

ASGAL è in grado di rilevarli tutti tranne il caso 2.

\newpage

\begin{figure}[t!]
	\centering
	\includegraphics[height=10cm,width=10cm]{images/alternativesplicingevents.jpg}
  \caption{I diversi tipi di Alternative Splicing}
  \label{fig:AlternativeSplicingTypes}
\end{figure}

\subsection{Paired-End Reads}
Le paired-end reads consistono nell'estrazione di due letture da un singolo frammento di DNA, contrariamente alle single-end reads che ne estraggono solo una. Sono prodotte da sistemi NGS, e la loro preparazione è molto semplice: una volta stabilita la grandezza della singola lettura, viene estratta la lettura sull'estremità sinistra, il campione viene girato, e viene estratta nuovamente l'estremità sinistra (ottenendo quindi l'estremità destra); viene generalmente fornita anche la distanza tra le due letture.

\begin{figure}[h!]
	\centering
	\includegraphics{images/pairedendreads2.png}
  \caption{Read paired-end}
  \label{fig:PairedEndReads}
\end{figure}
    	\subsection{Paired-End Reads}
Le paired-end reads consistono nell'estrazione di due letture da un singolo frammento di DNA (generalmente le due estremità), contrariamente alle single-end reads che ne estraggono solo una. Sono prodotte da sistemi NGS, e la loro preparazione è molto semplice: una volta stabilita la grandezza della singola lettura, viene estratta la lettura sull'estremità sinistra, il campione viene girato, e viene estratta nuovamente l'estremità sinistra (ottenendo quindi l'estremità destra). Viene inoltre fornita la distanza tra le due letture, che permette di disambiguare alcuni casi di allineamento.

\begin{figure}[b!]
	\centering
	\includegraphics[height=15cm,width=10cm]{images/pairedendreads.png}
  \caption{Esempi di read single-end e paired-end}
  \label{fig:AlternativeSplicingTypes}
\end{figure}
    	
%    \section{Modifiche effettuate ad ASGAL}
    	  
    	\section{Modifiche allo Splice-Aware Aligner}
Lo Splice-Aware Aligner svolge due compiti:
\begin{enumerate}
	\item Generazione dello Splicing Graph e della sua Linearizzazione
	\item Allineamento delle read di RNA-Seq alla Linearizzazione dello Splicing Graph
\end{enumerate}
L'allineamento avviene utilizzando il concetto di MEM (Maximum Exact Matching); l'output verrà convertito in formato SAM per permettere l'eventuale elaborazione con altri strumenti.

\subsection{Allineamento di entrambe le read}
Il primo problema da affrontare è ovviamente il fatto che sia ora necessario allineare due read e non una; fortunatamente si tratta solo di iterare il processo di allineamento su una coppia di read ad ogni ciclo, anziché su read singola. Verranno quindi generati due file contenti MEM anziché uno. Sarà poi compito della Formattazione SAM "fondere" i due file MEM per ottenere un SAM Completo.

\subsection{Introduzione di read unmapped e "placeholder" nel formato MEM}
Nei file MEM ottenuti dallo Splice-Aware Aligner vengono ora visuallizati due nuovi tipi di MEM: quelli relativi alle read unmapped e quelli relativi ai "placeholder". Il primo caso è banale, e rappresenta tutte quelle read che non hanno un matching esatto di lunghezza considerevole con il genoma dato in input.
Il secondo caso è più complesso e rappresenta un insieme di read fasulle utilizzate solo come padding per avere due file MEM della stessa lunghezza: questo facilita enormemente l'elaborazione nello step successivo (la formattazione SAM). Come detto in precedenza quando si lavora con read paired-end è sempre necessario lavorare a coppie, ma non sempre ad uno stesso pair è associato lo stesso numero di allineamenti secondari: è qui che entrano in gioco i "placeholder". La loro implementazione è banale: si tengono due contatori (che rappresentano rispettivamente il numero di allineamenti relativi alla prima read e quelli relativi alla seconda read), si ottengono separatamente gli allineamenti relativi a ciascuna delle due read, e si controllano i contatori. Si prende il minore dei due e si aggiungono tanti placeholder quanto bastano per rendere uguali i contatori.

\begin{algorithm}
\caption{Algoritmo per l'aggiunta dei placeholder}\label{placeholder}
\begin{algorithmic}[1]
%\Procedure{addPlaceholders}{}
	\State $count\_aligns\_1 \gets 0, count\_aligns\_2 \gets 0$
	\While{$count\_aligns\_1 != count\_aligns\_2$}
		\If{$count\_aligns\_1 < count\_aligns\_2$}
			\State $addPlaceholder(file1)$
			\State $count\_aligns\_1++$
		\Else
			\State $addPlaceholder(file2)$
			\State $count\_aligns\_2++$
		\EndIf
	\EndWhile\label{euclidendwhile}
	%\EndProcedure
\end{algorithmic}
\end{algorithm}

\begin{figure}[h]
	\centering
	\includegraphics[width=\linewidth]{images/tipiMEM2.png}
  \caption{I diversi tipi di MEM}
  \label{fig:AlternativeSplicingTypes}
\end{figure}

\subsection{Supporto alle fragment library types}


    	\section{Conversione in formato SAM}

\subsection{Descrizione generale}
Come già detto, gli allineamenti ottenuti dallo Splice-Aware aligner sono in un formato non standard chiamato MEM, che contiene solo informazioni relative ai MEMs ottenuti in fase di allineamento. L'obiettivo di questa seconda parte è quello di convertire i due file MEM in un singolo file SAM\footnote{\url{http://samtools.github.io/hts-specs/SAMv1.pdf}} (Sequence Alignment Map), il formato standard per memorizzare gli allineamenti.

Il formato SAM è composto da 11 campi:

\begin{enumerate}
	\item \textbf{QNAME}: Il nome identificativo dell'allineamento
	\item \textbf{FLAG}: Una serie di flag binari che identificano le caratteristiche dell'allineamento
	\item \textbf{RNAME}: Identificativo del gene di riferimento
	\item \textbf{POS}: Posizione 1-based di inizio dell'allineamento sul genoma
	\item \textbf{MAPQ}:  Valore che indica la qualità dell'allineamento
	\item \textbf{CIGAR}: Stringa che identifica le operazioni effettuate per ottenere l'allineamento
	\item \textbf{RNEXT}: QNAME del mate (solo paired-end)
	\item \textbf{PNEXT}: POS del mate	(solo paired-end)
	\item \textbf{TLEN}:  Distanza tra mate (solo paired-end)
	\item \textbf{SEQ}:  	Allineamento vero e proprio
	\item \textbf{QUAL}: Valore che indica la qualità delle read
\end{enumerate}

Non si tratta solo di una semplice conversione, in quanto è necessario indurre diverse informazioni aggiuntive per avere un file SAM standard, quali: la posizione di inizio dell'allineamento sulla genomica, la stringa CIGAR, i flag relativi all'allineamento, ecc. Per supportare le read paired-end è stato necessario modificare gran parte di queste funzionalità. In particolare, è stata modificato il modo in cui vengono calcolati i campi FLAG, RNEXT, PNEXT e TLEN.

\newpage

\subsection{Computazione del campo FLAG per read paired-end}
Il campo FLAG è il secondo del formato SAM e consiste di un valore numerico (ottenuto convertendo in decimale una serie di flag binari) che rappresenta le caratteristiche dell'allineamento preso in esame. La seguente immagine mostra il significato di ciascun bit del flag:

\begin{figure}[h]
	\centering
	\includegraphics[width=\linewidth]{images/samflag.png}
  \caption{Il significato di ciascun bit del campo FLAG }
  \label{fig:SAM Flags}
\end{figure}

Nei casi single-end solo due flag vengono utilizzati: quello relativo allo strand (0x16) e quello relativo al tipo di allineamento (0x100); visto che le read non sono paired, il flag 0x1 sarà sempre false, quindi tutti i flag risultanti saranno pari.

Nei casi paired-end tutti i flag vengono utilizzati. E' inoltre necessario trattare gli allineamenti a coppie, in quanto il campo FLAG esprime informazioni anche sul mate e non solo sull'allineamento preso in esame. 

Supponiamo ad esempio di avere due read, la prima che mappa sullo strand positivo e la seconda che non mappa (ed è quindi \textit{unmapped}). Sarà innanzitutto necessario mettere a true il flag relativo alle read paired-end (0x1) per entrambe le read. Considerando la prima, sarà messo a true il flag relativo al mate unmapped (0x8) e il flag relativo al first-in-pair (0x4). Considerando la seconda, sarà messo a true il flag relativo alla read unmapped (0x4) e quello relativo al second-in-pair(0x80). I flag in decimale saranno quindi 73 e 133.

Si noti che per il momento non viene tenuto conto dei flag 0x200 e 0x400, ma questo non è di alcuna rilevanza al fine di identificare eventi di Alternative Splicing.

\newpage

\subsection{Computazione dei campi RNEXT, PNEXT e TLEN}
I campi TLEN, RNEXT e PNEXT rappresentano rispettivamente il settimo, l'ottavo e il nono campo di ogni record del formato SAM; essi sono praticamente inutilizzati quando si allineano read single-end, ma nel formato paired-end assumono maggiore importanza. In particolare i campi RNEXT e PNEXT sono utilizzati da strumenti di visualizzazione degli allineamenti (come ad esempio IGV \cite{IGV}) per permettere una corretta visualizzazione di una read e del suo mate.

Il campo RNEXT contiene il nome dell'allineamento relativo al mate (ovvero il suo campo PNAME). Per semplicità, quando i due allineamenti sono consecutivi, si può lasciare il suo valore a '='.

Il campo PNEXT contiene la posizione iniziale 1-based dell'allineamento relativo al mate (ovvero il suo campo POS. Qualora il mate fosse unmapped, si utilizza il valore 0.

Il campo TLEN rappresenta la distanza la lunghezza del template osservato, ovvero la distanza (sul genoma) tra l'inizio della prima read e la fine della seconda. Per la sua computazione è sufficiente trovare la posizione finale del secondo allineamento e sottrarre la posizione iniziale del primo.

La seguente immagine mostra un esempio di allineamento visualizzato da IGV:

\begin{figure}[h]
	\centering
	\includegraphics[width=\linewidth]{images/mateinfo.png}
  \caption{Informazioni sull'allineamento}
  \label{fig:MateInfo}
\end{figure}

\newpage

E' importante notare che, se questi campi sono settati correttamente, lasciando il cursore del mouse su un allineamento vengono visualizzate tutte le informazioni relative al mate. Al contrario, se si prova ad indicizzare il file BAM (la versione binaria del formato SAM) per l'utilizzo con IGV, e questi campi non sono stati settati correttamente, verrà visualizzato un errore. Un esempio errore è il seguente: se il campo FLAG non contiene 0x8 (quindi il mate è mapped), e si inserisce un valore di PNEXT diverso da 0, al momento dell'indicizzazione sarà visualizzato il messaggio "mapped mate cannot have zero coordinate; treated as unmapped".

\subsection{Calcolo delle statistiche dell'allineamento}
Seguendo l'esempio di altri allineatori (come ad esempio STAR), è stato deciso di visualizzare alcune statistiche in fase di allineamento, quali:

\begin{itemize}
	\item Numero di read mappate, non mappate e "placeholder"
	\item Numero di allineamenti primari e secondari
	\item IDMP medio
	\item TIDMP medio
\end{itemize}

Questi valori sono visualizzati nel file \textit{.alignsinfo.txt}. Anche se non hanno finalità particolari per la rilevazione di eventi di Alternative Splicing, essi forniscono uno strumento per valutare la qualità degli allineamenti effettuati da ASGAL.

\makeatletter
\newenvironment{CenteredBox}{% 
\begin{Sbox}}{% Save the content in a box
\end{Sbox}\centerline{\parbox{\wd\@Sbox}{\TheSbox}}}% And output it centered
\makeatother

\begin{figure}[thp]
\begin{CenteredBox}
  \begin{lstlisting}
Count mapped1: 1281/1313
Count mapped2: 1291/1295
Count unmapped reads1: 32
Count unmapped reads2: 4
Count placeholders1: 69
Count placeholders2: 87
Count mapped pairs: 1194
Count primary alignments: 958
Count secondary alignments: 236
Count one-side alignments: 184
Average idmp: 253.75795644891122
Average tidmp: 141.38358458961474
Average tlen: 915.1889352818372
  \end{lstlisting}
\end{CenteredBox}
\caption{Esempio di file \textit{.alignsinfo.txt}}
\end{figure}

%\begin{figure}[h!]
%	\centering
%	\includegraphics[height=5.5cm,width=8cm]{images/aligninfo.png}
%  \caption{Esempio di file \textit{.alignsinfo.txt}}
%  \label{fig:AlignsInfo}
%\end{figure}
    	\section{Rilevatore di Eventi di Alternative Splicing}

\subsection{Cos'è e come funziona}

\newpage

\subsection{Fusione degli introni dedotti dai due sample}

\newpage

\subsection{Calcolo dell' IDMP (Inner Distance between Mate Pairs)}
Considerando una coppia di read, si definisce IDMP (Inner Distance between Mate Pairs) la distanza sul genoma di riferimento in termini di BP (Base Pair) tra l'ultima base della prima read e la prima della seconda. Questa informazione viene generalmente fornita dall'ente che ha effettuato il sequenziamento, e può essere confrontata con l'IDMP rilevato durante l'allineamento per rilevare nuovi eventi di Alternative Splicing.

Visto che un allineamento può essere rappresentato da più di un MEM, non è possibile semplicemente aggiungere la lunghezza dell'allineamento alla sua posizione iniziale. Prima di poter calcolare l'IDMP è quindi necessario introdurre il concetto di BitVector, ovvero una sequenza di bit che rappresenta la posizione degli esoni nella genomica. Un BitVector è dotato di due operazioni:

\begin{itemize}
	\item Rank: data una posizione, ritorna l'esone di provenienza
	\item Select: dato un esone, ritorna la posizione di partenza 
\end{itemize}

Queste due operazioni permettono di calcolare l'IDMP in maniera efficace. Innanzitutto si prende l'ultimo MEM relativo all'allineamento della prima read, e si utilizza l'operazione di Rank per trovare l'esone di appartenenza. A questo punto, si utilizza l'operazione di Rank per trovare la posizione iniziale dell'esone. L'offset sarà quindi dato dalla differenza tra il MEM e la posizione iniziale dell'esone. Basta quindi aggiungere questo offset alla posizione iniziale per trovare la fine del primo allineamento.

Il seguente algoritmo riassume la procedura:

\newpage

\subsection{Calcolo del TIDMP (Transcript-based IDMP)}
Per TIDMP si intende la misura della distanza \textit{sui trascritti} tra le due read. Per il momento viene calcolata solo su esoni consecutivi.

\newpage

\subsection{Possibile utilizzo di IDMP e TIDMP}

\newpage
    	
    	\section{Esempio di funzionamento}

In questo esempio di funzionamento si utilizzerà il gene ENSG00000280145, situato nel cromosoma 21 dell'uomo sapiens (GRCh38 / hg38). E' stato prima scaricato il   genoma di riferimento da ensembl in formato fasta e la relativa annotazione in formato gtf. Dal file gtf (contenente l'annotazione per l'intero genoma) è stata isolata l'annotazione relativa al gene ENSG00000280145.

\subsection{Generazione delle read}

Si è scelto di utilizzare Flux Simulator per la generazione delle read paired-end. Il suo utilizzo non è particolarmente complicato, ma è necessario passare i diversi parametri attraverso un file con estensione .p. Il file utilizzato in questa simulazione è il seguente:

\begin{figure}[h]
	\centering
	\includegraphics[width=\linewidth]{images/parameters.png}
  \caption{Il file contente i parametri di Flux Simulator}
  \label{fig:Parameters}
\end{figure}

Vengono così generati due file contenti 2000 read di lunghezza 100, in formato fasta, che saranno dati in input ad ASGAL.

\newpage

\subsection{Utilizzo}

ASGAL viene eseguito via linea di comando, richiamando lo script principale usando come parametri:

\begin{itemize}
	\item Il genoma di riferimento (opzione -g)
	\item L'annotazione del genoma (opzione -a)
	\item I due file contenti read (opzioni -s e -s2)
	\item La cartella di destinazione dell'output (opzione -o)
	\item L'indicazione delle read paired-end (opzione --paired)
	\item La fragment library type (opzione -f), opzionale per velocizzare la fase di allineamento
\end{itemize}

Questo script richiama nell'ordine lo Splice-Aware Aligner, il Formattatore SAM e il Rilevatore di eventi di Alternative Splicing, visualizzando alcune informazioni sul funzionamento.

Questa immagine mostra il funzionamento di ASGAL:

\begin{figure}[h]
	\centering
	\includegraphics[width=\linewidth]{images/prompt.png}
  \caption{ASGAL in funzione}
  \label{fig:Parameters}
\end{figure}

Sebbene sia possibile eseguire ciascuno script singolarmente, si raccomanda di usare lo script principale per un utilizzo più immediato. 

\newpage

\subsection{Risultati}

Sono stati rilevati i seguenti eventi di Alternative Splicing:

\begin{figure}[h]
	\centering
	\includegraphics[width=\linewidth]{images/results.png}
  \caption{Eventi di Alternative Splicing rilevati da ASGAL}
  \label{fig:Parameters}
\end{figure}

\newpage
    	
		\section{Conclusioni}
ASGAL è ora in grado di supportare le read paired-end, rilevando correttamente gli eventi di Alternative Splicing e salvando gli allineamenti rispettando lo standard del formato SAM. Rimane da investigare l'utilizzo di IDMP e TIDMP per aumentare l'efficacia della rilevazione.

\end{document}
