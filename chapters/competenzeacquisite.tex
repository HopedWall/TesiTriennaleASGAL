\section{Competenze acquisite}
Durante lo svolgimento dello stage sono state acquisite le seguenti competenze:

\begin{itemize}
	\item Utilizzo di dati biologici in formati diversi (fasta, gtf, SAM, ecc.) e creazione di algoritmi che li manipolano
	\item Utilizzo di strumenti di natura bioinformatica: è stato utilizzato SAMTools per la validazione dei file SAM, IGV per la visualizzazione degli allineamenti sul genoma, gli allineatori BWA, STAR e BBMap come riferimento per lo Splice-Aware Aligner, RNASeqSim e Flux Simulator per la generazione di read
	\item Utilizzo di community di esperti di bioinformatica: è stato usato Biostars per chiarire alcuni dubbi inerenti al funzionamento di SAMTools, il thread si trova a questo indirizzo: https://www.biostars.org/p/376192/
	\item Approfondimento dei linguaggi di programmazione Python e C++ e utilizzo di librerie specifiche per la bioinformatica (come ad esempio kseq per il parsing di file fasta)
	\item Utilizzo avanzato di Linux e di strumenti di environment management specifici per la bioinformatica (Bioconda)
	\item Utilizzo di strumenti di version control: è stato utilizzato Github, la relativa repo si trova a questo indirizzo: https://github.com/HopedWall/galig
	\item Lavoro di gruppo: è stato utilizzato il software Slack per comunicare con i membri dell'Algolab
\end{itemize}