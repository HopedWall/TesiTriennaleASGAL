\section{Rilevatore di Eventi di Alternative Splicing}

\subsection{Cos'è e come funziona}

\newpage

\subsection{Fusione degli introni dedotti dai due sample}

Come già detto in precedenza, i nuovi introni rilevati dagli allineamenti vengono estratti dal file MEM e confrontati con quelli già noti al fine di rilevare eventi di Alternative Splicing. In questo senso, la versione paired rileva i nuovi introni da ciascuno dei due file MEM, e li fonde.

%Rivedere

\newpage

\subsection{Calcolo dell' IDMP (Inner Distance between Mate Pairs)}
Considerando una coppia di read, si definisce IDMP (Inner Distance between Mate Pairs) la distanza sul genoma di riferimento in termini di BP (Base Pair) tra l'ultima base della prima read e la prima della seconda. Questa informazione viene generalmente fornita dall'ente che ha effettuato il sequenziamento, e può essere confrontata con l'IDMP rilevato durante l'allineamento per rilevare nuovi eventi di Alternative Splicing.

Visto che un allineamento può essere rappresentato da più di un MEM, non è possibile semplicemente aggiungere la lunghezza dell'allineamento alla sua posizione iniziale. Prima di poter calcolare l'IDMP è quindi necessario introdurre il concetto di BitVector, ovvero una sequenza di bit che rappresenta la posizione degli esoni nella genomica. Un BitVector è dotato di due operazioni:

\begin{itemize}
	\item Rank: data una posizione, ritorna l'esone di provenienza
	\item Select: dato un esone, ritorna la posizione di partenza 
\end{itemize}

Queste due operazioni permettono di calcolare l'IDMP in maniera efficace. Innanzitutto si prende l'ultimo MEM relativo all'allineamento della prima read, e si utilizza l'operazione di Rank per trovare l'esone di appartenenza. A questo punto, si utilizza l'operazione di Rank per trovare la posizione iniziale dell'esone. L'offset sarà quindi dato dalla differenza tra il MEM e la posizione iniziale dell'esone. Basta quindi aggiungere questo offset alla posizione iniziale per trovare la fine del primo allineamento.

% Sistemare, spiegare come viene calcolato.
L'inizio del secondo allineamento è noto. A questo punto basterà sottrarre ad esso la fine di quello precedente, in modo da ottenere l'IDMP.

Il seguente algoritmo riassume la procedura:

\newpage

\subsection{Calcolo del TIDMP (Transcript-based IDMP)}
Per TIDMP si intende la misura della distanza \textit{sui trascritti} tra le due read. Per calcolarlo è innanzitutto necessario ottenere l'ultimo MEM relativo alla prima read e il primo MEM relativo alla seconda. Da ciascuno di essi è possibile ottenere la posizione iniziale e la lunghezza sul bitvector. A questo punto, usando l'operazione di rank, è possibile capire l'esone di provenienza di ciascun MEM.

Se i due MEM si trovano sullo stesso esone, il TIDMP è dato semplicemente dalla distanza tra la fine del primo MEM e l'inizio del secondo (sarà quindi uguale all' IDMP calcolato precedentemente).

Se i due MEM non si trovano sullo stesso esone, potrebbero essere su due esoni consecutivi o meno. Nel caso di esoni consecutivi, il TIDMP è dato dalla somma tra il suffisso non coperto dal primo esone e il prefisso non coperto dal secondo esone.

Il seguente algoritmo mostra come calcolare il TIDMP:



Il caso di esoni non consecutivi presenta diverse criticità, e per il momento non è stato calcolato.

\newpage

\subsection{Possibile utilizzo di IDMP e TIDMP}

\newpage