\section{Rilevatore di Eventi di Alternative Splicing}

\subsection{Cos'è e come funziona}

\newpage

\subsection{Fusione degli introni dedotti dai due sample}

\newpage

\subsection{Calcolo dell' IDMP (Inner Distance between Mate Pairs)}
Considerando una coppia di read, si definisce IDMP (Inner Distance between Mate Pairs) la distanza sul genoma di riferimento in termini di BP (Base Pair) tra l'ultima base della prima read e la prima della seconda. Questa informazione viene generalmente fornita dall'ente che ha effettuato il sequenziamento, e può essere confrontata con l'IDMP rilevato durante l'allineamento per rilevare nuovi eventi di Alternative Splicing.

Visto che un allineamento può essere rappresentato da più di un MEM, non è possibile semplicemente aggiungere la lunghezza dell'allineamento alla sua posizione iniziale. Prima di poter calcolare l'IDMP è quindi necessario introdurre il concetto di BitVector, ovvero una sequenza di bit che rappresenta la posizione degli esoni nella genomica. Un BitVector è dotato di due operazioni:

\begin{itemize}
	\item Rank: data una posizione, ritorna l'esone di provenienza
	\item Select: dato un esone, ritorna la posizione di partenza 
\end{itemize}

Queste due operazioni permettono di calcolare l'IDMP in maniera efficace. Innanzitutto si prende l'ultimo MEM relativo all'allineamento della prima read, e si utilizza l'operazione di Rank per trovare l'esone di appartenenza. A questo punto, si utilizza l'operazione di Rank per trovare la posizione iniziale dell'esone. L'offset sarà quindi dato dalla differenza tra il MEM e la posizione iniziale dell'esone. Basta quindi aggiungere questo offset alla posizione iniziale per trovare la fine del primo allineamento.

Il seguente algoritmo riassume la procedura:

\newpage

\subsection{Calcolo del TIDMP (Transcript-based IDMP)}
Per TIDMP si intende la misura della distanza \textit{sui trascritti} tra le due read. Per il momento viene calcolata solo su esoni consecutivi.

\newpage

\subsection{Possibile utilizzo di IDMP e TIDMP}

\newpage