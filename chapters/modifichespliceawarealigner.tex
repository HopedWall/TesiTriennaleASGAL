\section{Generazione dello Splicing Graph e allineamento Splice-Aware}

\subsection{Descrizione generale}

Il primo compito di ASGAL è quello di generare lo Splicing Graph a partire dal genoma di riferimento (in formato FASTA\footnote{\url{https://blast.ncbi.nlm.nih.gov/Blast.cgi?CMD=Web&PAGE_TYPE=BlastDocs&DOC_TYPE=BlastHelp}}) e dalla sua annotazione (in formato GTF\footnote{\url{https://www.ensembl.org/info/website/upload/gff.html}}). Uno splicing graph è un grafo orientato in cui i nodi rappresentano gli esoni, e gli archi collegano due esoni che compaiono uno dopo l'altro in un trascritto; utilizzando la chiusura transitiva ogni esone viene poi collegato a tutti i suoi successori.

Per permettere un allineamento tra due stringhe viene generata una linearizzazione dello Splicing Graph, ottenuta semplicemente concatenandone i nodi e utilizzando un carattere speciale come delimitatore tra un nodo e un altro. 

Questa prima fase non utilizza le read paired-end, e non è stata quindi modificata nell'adattamento al nuovo tipo di read.

\begin{figure}[h!]
	\centering
	\includegraphics{images/splicinggraph.png}
  \caption{Un esempio di Splicing Graph}
  \label{fig:PairedEndReads}
\end{figure}

Il secondo compito di ASGAL è quello di allineare le read paired-end con la linearizzazione appena ottenuta. Per fare questo, utilizza il concetto di MEM (Maximum Exact Matches), ovvero di una tripla così costituita:
\begin{enumerate}
	\item Posizione iniziale nel genoma
	\item Posizione iniziale nella read
	\item Lunghezza della parte in comune
\end{enumerate}
	
Come si può intuire dal nome, i MEM utilizzano il concetto di pattern matching esatto, rispetto  ai generici algoritmi di allineamento che ammettono dei mismatch o degli indel; ad ogni allineamento corrispondono uno o più MEM.

Il nuovo formato paired-end è stato qui utilizzato per velocizzare il processo di allineamento ove possibile. Al momento non sono state utilizzate per migliorare l'efficacia dell'allineamento; si rimanda alla sezione "Sviluppi futuri" per informazioni su come questo potrebbe accadere.

\newpage

Il risultato dell'allineamento sarà restituito, per comodità, nel formato \textit{.mem} così composto:

\begin{itemize}
	\item Lo \textbf{strand} dell'allineamento
	\item	Il \textbf{nome della read} allineata
	\item Il \textbf{numero di errori} commessi nell'allineamento 
	\item Il \textbf{MEM} vero e proprio, eventualmente anche più di uno
	\item La \textbf{sottostringa della read} che costituisce il MEM considerato
\end{itemize}

\subsection{Allineamento di entrambe le read}
%Il primo problema da affrontare è ovviamente il fatto che sia ora necessario allineare due read per ciclo e non una; fortunatamente si tratta solo di iterare il processo di allineamento su una coppia di read ad ogni ciclo, anziché su read singola. Verranno quindi generati due file contenti MEM anziché uno. Sarà poi compito della Formattazione SAM "fondere" i due file MEM per ottenere un SAM Completo.
La prima differenza sostanziale rispetto al caso single-end è che nel caso paired-end le read da allineare sono sostanzialmente due (ovvero le due estremità); bisogna inoltre tenere conto della distanza tra di esse. 

Mentre in un generico allineatore questo avrebbe creato non pochi problemi, in ASGAL la soluzione è banale: basta trattare le due estremità come read singole, e allinearle indipendentemente. Questo è possibile in quanto ciò che si vuole ottenere è un insieme di MEM, e non siamo interessati all'ottimizzazione globale dell'allineamento. %Rivedere?

Resta comunque fondamentale riuscire a collegare i MEM delle due estremità tra di loro. Per fare questo, abbiamo deciso di utilizzare due file .mem anziché uno, con la seguente proprietà: una stessa riga nel primo file e nel secondo corrispondono alla stessa read. Questo risolve il problema del collegamento tra MEM provenienti dalla stessa read, e indirettamente anche quello della distanza tra le estremità (dati i MEM è possibile calcolare la distanza).

Non è però detto che entrambe le estremità siano coinvolte nello stesso numero di allineamenti (potrebbero addirittura non esserlo affatto). E' quindi necessario introdurre nuovi tipi di MEM per questi casi.


\subsection{Introduzione di read unmapped e placeholder nel formato mem}
Nei file mem ottenuti dallo Splice-Aware Aligner vengono ora visuallizati due nuovi tipi di MEM: quelli relativi alle read unmapped e quelli relativi ai placeholder. Il primo caso è banale, e rappresenta tutte quelle read che non hanno un matching esatto di lunghezza considerevole con il genoma dato in input.

Il secondo caso è più complesso e rappresenta un insieme di read fasulle utilizzate solo come padding per avere due file MEM della stessa lunghezza: questo facilita enormemente l'elaborazione nello step successivo (la formattazione SAM). Come detto in precedenza quando si lavora con read paired-end è sempre necessario lavorare a coppie, ma non sempre ad uno stesso pair è associato lo stesso numero di allineamenti secondari: è qui che entrano in gioco i placeholder. 

I placeholder sono stati implementati in questo modo: si tengono due contatori (che rappresentano rispettivamente il numero di allineamenti relativi alla prima read e quelli relativi alla seconda read), si ottengono separatamente gli allineamenti relativi a ciascuna delle due read, e si controllano i contatori. Si prende il minore dei due e si aggiungono tanti placeholder quanto bastano per rendere uguali i contatori.

Il seguente frammento di codice mostra la procedura:

\begin{lstlisting}[language=C++]
        //head_1 and head_2 are the IDs of the alignments
        if (count1 < count2) {
            while (count1 < count2) {
                outFile_1 << "PLACEHOLDER" << " " << head_1 << "\n";
                count1++;
            }
        } else if (count2 < count1) {
            while (count2 < count1) {
                 outFile_2 << "PLACEHOLDER" << " " << head_2 << "\n";
                 count2++;
            }
        }
\end{lstlisting}


E' stato inoltre necessario introdurre un nuovo campo nel formato \textit{.mem}, che permette di distinguere il tipo di MEM preso in esame: ad esempio trovando come primo valore la stringa "MAPPED", si capisce che il MEM preso in esame è relativo ad una read che è stata mappata correttamente; ovviamente vale lo stesso ragionamento con le stringhe "UNMAPPED" e "PLACEHOLDER".

L'immagine seguente mostra un esempio di file nel nuovo formato mem:

\begin{figure}[h!]
	\centering
	\includegraphics[width=\linewidth]{images/tipiMEM4.png}
  \caption{Esempio di file in formato \textit{.mem}}
  \label{fig:MEMTypes}
\end{figure}

\newpage

\subsection{Supporto alle fragment library types}
Ci sono diversi protocolli per la preparazione di librerie paired-end, che portano a read con caratteristiche diverse. Le \textbf{fragment library types}\footnote{\url{https://salmon.readthedocs.io/en/latest/}} permettono di descrivere queste caratteristiche in modo sintetico. Le caratteristiche che possono essere descritte sono:
\begin{itemize}
	\item \textbf{Orientamento relativo di una read rispetto all'altra}: può essere inward (I) o outward (O)
	\item \textbf{Se è noto o meno lo strand di appartenenza delle due read}:  può essere stranded (S) o unstranded (U)
	\item \textbf{La direzionalità della prima read, solo nel caso stranded}: può essere first (F) o reverse (R)
\end{itemize}

La seguente immagine descrive tutte le possibili combinazioni:
\begin{figure}[h!]
	\centering
	\includegraphics[width=\linewidth, height=10cm]{images/fragmentlibrarytypes.png}
  \caption{I diversi tipi di FTL}
  \label{fig:FragmentLibraryTypes}
\end{figure}

\newpage

ASGAL utilizza queste informazioni per velocizzare il processo di allineamento. Nel formato single-end questa informazioni non esiste, quindi ogni read veniva allineata in entrambe le direzioni, e si prendeva l'allineamento migliore dei due. 

Nel formato paired-end, si può utilizzare la ftl per ridurre il numero di allineamenti necessari, fino al 50\% nel caso stranded. Ad esempio, se viene fornita un ftl di tipo ISF è sufficiente allineare la read 1 sullo strand + (ignorando lo strand -) e la read 2 sullo strand - (ignorando lo strand +).

Il caso unstranded richiede un po' più di attenzione: non è infatti possibile sapere a priori su quale strand allineare la prima read. Per risolvere questo problema è stata riutilizzata la vecchia procedura di allineamento della prima read in entrambe le direzioni; una volta trovata la direzione migliore per la prima, la seconda viene di conseguenza. 

Supponiamo ad esempio di avere una libreria in formato MU: se la prima read allinea sullo strand +, di conseguenza anche la seconda sarà allineata sullo strand +; viceversa, se la prima read allinea sullo strand -, anche la seconda allinea sempre sullo strand -. In questo caso si ottiene solo una riduzione del 25\% nel numero degli allineamenti.

Qualora il tipo di libreria non venga fornito dall'utente, rimane necessario provare ad allineare le read in entrambe le direzioni, senza alcuni incremento di prestazioni.


