% Abstract 
\renewcommand{\abstractname}{Abstract - Italiano}
\begin{abstract}
L'Alternative Splicing è un meccanismo attraverso il quale diverse proteine sono generate a partire da uno stesso gene. Si stima che oltre il 75\% dei geni umani utilizzi l' Alternative Splicing, e una piena comprensione di questo meccanismo potrebbe aiutare a far luce su diversi fenomeni biologici non ancora del tutto chiari, oltre che a migliorare la capacità di rilevazione di diverse patologie di natura genetica. Con l'avvento delle tecnologie NGS (Next Generation Sequencing), l'accesso a grandi quantità di informazioni di natura biologica è diventato sempre più facile e conveniente: in questo contesto l'informatica gioca un ruolo fondamentale nello studio dell'Alternative Splicing. Le read paired-end, prodotte da sequenziatori NGS, sono ad oggi molto utilizzate per migliorare la qualità degli allineamenti rispetto alle read single-end. ASGAL (Alternative Splicing Graph ALigner) è un software sviluppato dall'Algolab in grado di rilevare eventi di Alternative Splicing espressi in un campione di RNA-Seq a partire da un genoma di riferimento e dall'annotazione di un gene. Al momento ASGAL non supporta le read paired-end, che potrebbero migliorarne le capacità di rilevazione. Abbiamo quindi deciso di aggiungere il supporto alle read paired-end. In questo documento saranno evidenziate le principali modifiche apportate ad esso apportate, ponendo l'attenzione sulle differenze tra il formato single-end e quello paired-end. Sarà poi presentato un esempio di funzionamento del suo funzionamento, oltre che ad alcuni possibili sviluppi futuri.
   \end{abstract}
   
	\renewcommand{\abstractname}{Abstract - English}   
   \begin{abstract}
Alternative Splicing is a mechanism by which different proteins are produced starting from the same gene. It is estimated that over 75\% of human genes undergo Alternative Splicing, and a full comprehension of such mechanism could help shed light on different biological phenomena which are not fully understood yet, and also to improve the ability to detect genetic diseases. With the advent of NGS (Next Generation Sequencing) technologies, access to biological data has become easier and cheaper: in this context computer science plays a key role in the study of Alternative Splicing. Paired-end reads, produced by NGS sequencers, are nowadays widely used to improve the quality of the alignments compared to single-end reads. ASGAL (Alternative Splicing Graph ALigner) is a software developed by Algolab capable of detecting Alternative Splicing events expressed in an RNA-Seq sample starting from a reference genome and a gene annotation. At the moment ASGAL does not support paired-end reads, which could lead to an improvement to its event detection capabilities. Therefore, we have decided to add paired-end support. In this paper the main changes will be highlited, focusing our attention on differences between single-end and paired-end formats. An example of its execution will be presented, as well as some future prospects.
   \end{abstract}